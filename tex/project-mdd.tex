\documentclass[12pt]{report}

\usepackage[a4paper, total={17cm, 24cm}]{geometry}
\usepackage{exercise}
\usepackage{tikz}

\usepackage{amsmath}
\usepackage{amssymb}
\usepackage{amsfonts}
\usepackage{stmaryrd}


\renewcommand{\ExerciseHeader}{\noindent\textbf{\large\ExerciseName\ %
\ExerciseHeaderNB\ExerciseHeaderTitle
\ExerciseHeaderOrigin\medskip}}
\setlength{\QuestionIndent}{1.5em}

\newcommand{\answerbox}[2]{\hfill\break\\
        \framebox[\linewidth]{\parbox[c][#1][c]{\dimexpr\linewidth-2\fboxsep-2\fboxrule}{#2}}
}

\renewcommand{\arraystretch}{1.2} % vertical padding for tabular environment

\begin{document}

\hfill
\begingroup
\Large
\begin{tabular}{|l|p{6cm}|}
	\hline
	First \& last name &
	% YOUR NAME HERE
	\\ \hline
	NOMA UCLouvain & 
	% YOUR NOMA HERE
	\\ \hline
\end{tabular}
\endgroup
\vspace{1.5cm}

\noindent
\begingroup
	\Large
	\textbf{LINFO2266: Advanced Algorithms for Optimization}\\\\
	Project 6: MDD
\endgroup
\vspace{0.2cm}

%\begin{Exercise}[title={Implementing the Solver}]
%\\
%The first task of this assignment is for you to implement the vanilla
%algorithm as it has been explained during the lecture (see slide 67). To help you, we 
%provide you with a well-documented framework that defines and implements 
%all the abstractions you will need to implement a generic solver. 
%Additionally, and because the BaB-MDD framework parallelizes \emph{VERY} 
%well, we provide you with a parallel implementation of the algorithm
%(ParallelSolver). Digging into that code, understanding it, and 
%stripping away all the parallel-related concerns should finalize to give 
%you a thorough understanding of the sequential algorithm.  
%Before diving into the code, answer the following few questions to make sure you get 
%a good grasp of the basics behind BaB-MDD algorithm.
%\end{Exercise}
%
%\begin{Exercise}[title={Definitions: Exact, Relaxed, and Restricted MDD }]
%\Question Give the formal definition of an \emph{exact MDD} encoding the problem $\mathcal{P}$
%\answerbox{1.5cm}{
%% YOUR SOLUTION HERE
%}
%
%\Question Give the formal definition of a \emph{restricted MDD} for the problem $\mathcal{P}$
%\answerbox{1.5cm}{
%% YOUR SOLUTION HERE
%}
%
%\Question Give the formal definition of a \emph{relaxed MDD} for the problem $\mathcal{P}$
%\answerbox{1.5cm}{
%% YOUR SOLUTION HERE
%}
%\end{Exercise}
%
%\newpage
%
\begin{Exercise}[title={Application to Knapsack}]\\
Consider the following figure: it depicts an exact DD encoding all solutions to a knapsack problem instance (it uses the same model as in the slides). In this instance, the sack has a capacity of $70$, and the costs-and-profits of each item are given as follows:
\begin{center}
\begin{tabular}{| l | r | r | r | r | }
	\hline
	                            &$x_0$ &$x_1$&$x_2$ &$x_3$ \\
	\hline
	weight ($w_i$) &       20 &     20 &      25  &       30 \\
	profit    ($p_i$) &      15  &      15 &       40 &       35 \\
	\hline
\end{tabular}

\vspace*{1.5cm}
\includegraphics*[width=.9\textwidth]{kp_exact.png}
\end{center}

\vspace*{1.5cm}
\Question What is the best solution encoded in this DD, and what is its objective value ?
\answerbox{3cm}{
% YOUR SOLUTION HERE
}

\newpage
\Question Draw the relaxed DD corresponding to the example instance with $W = 3$ (Your selection heuristic must decide to merge the nodes with lowest remaining capacity). 
\answerbox{20cm}{
% YOUR SOLUTION HERE
}

\newpage
\Question Draw the restricted DD corresponding to the example instance with $W = 3$  (Your selection heuristic must decide to delete the nodes with lowest remaining capacity). 
\answerbox{20cm}{
% YOUR SOLUTION HERE
}
\end{Exercise}

\newpage
\begin{Exercise}[title={Building Nuclear Plants}]
\\
To decarbonate the European economy, Mrs. From-the-Streets, minister of energy, 
has decided to refresh electric infrastructure and build new nuclear plants all over the
continent. This, however, has to be done in a principled way and no two power plants can
be located too close to one another as there is a risk it would cause excess tension.
Given a map of the infrastructure with candidate build sites, you are asked to find 
the set of locations where to construct power plants without violating safety constraints.

In practice, the map will be given to you in the form of a sizeable \emph{undirected} graph. 
In that graph, each node represents a candidate build site, and two sites are connected
with an edge if they are too close to one another (or more generally, if building a power
plant at both sites causes a potential safety hazard). The problem which you are asked to 
solve is thus the following: you must find a maximum subset of the nodes in the graph
such that no two construction sites are connected with an edge in the underlying graph.

\subsection*{Example:}
The below graph depicts a situation where 6 candidate construction sites have been identified.
It also shows that it would not be safe to build a nuclear plant at site 1 and 2 at the same
time, or to build some at both sites 3 and 5. In this example, a solution that maximizes the
energy production while guaranteeing safety constraints is the set $\left\{ 1, 4, 5, 6 \right\}$.

\begin{center}
\begin{tikzpicture}[
	  nuke/.style = {
		circle, 
		minimum size=#1,
		text height=1.7*#1,
		append after command={%
		  \pgfextra{ 
			\foreach \ang in {0,120,240}
			\draw[rotate around={\ang:(0,0)}] (\tikzlastnode.center) ellipse (0.45*#1 and 0.15*#1); 
			\fill (\tikzlastnode.center) circle (0.05*#1);
		  }
		}
	  }
	]
	
	\node [nuke=10mm] (1) at (0,0)    {1};
	\node [nuke=10mm] (2) at (5,0)    {2};
	\node [nuke=10mm] (3) at (5,-5)   {3};
	\node [nuke=10mm] (4) at (0,-5)   {4};
	\node [nuke=10mm] (5) at (10,0)   {5};
	\node [nuke=10mm] (6) at (10,-5)  {6};
	
	\draw[-] (1) -- (2);
	\draw[-] (2) -- (3);
	\draw[-] (2) -- (4);
	\draw[-] (2) -- (5);
	\draw[-] (3) -- (4);
	\draw[-] (3) -- (5);
	\draw[-] (3) -- (6);
\end{tikzpicture}
\end{center}

\Question What information is contained in the states of the model?
\answerbox{4cm}{
% YOUR SOLUTION HERE
}

\Question Give the transition function of your problem
\answerbox{3cm}{
% YOUR SOLUTION HERE
}

\Question Give the transition cost function of your problem.
\answerbox{3cm}{
% YOUR SOLUTION HERE
}

\Question Give the definition of the relaxation operator $\oplus$ which you will be using.
\answerbox{3cm}{
% YOUR SOLUTION HERE
}
\end{Exercise}

\end{document}