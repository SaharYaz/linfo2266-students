\documentclass[12pt]{report}

\usepackage[a4paper, total={17cm, 24cm}]{geometry}
\usepackage{exercise}

\renewcommand{\ExerciseHeader}{\noindent\textbf{\large\ExerciseName\ %
\ExerciseHeaderNB\ExerciseHeaderTitle
\ExerciseHeaderOrigin\medskip}}
\setlength{\QuestionIndent}{1.5em}

\newcommand{\answerbox}[2]{\hfill\break\\
        \framebox[\linewidth]{\parbox[c][#1][c]{\dimexpr\linewidth-2\fboxsep-2\fboxrule}{#2}}
}

\renewcommand{\arraystretch}{1.2} % vertical padding for tabular environment

\begin{document}

\hfill
\begingroup
\Large
\begin{tabular}{|l|p{6cm}|}
	\hline
	First \& last name &
	% YOUR NAME HERE
	\\ \hline
	NOMA UCLouvain & 
	% YOUR NOMA HERE
	\\ \hline
\end{tabular}
\endgroup
\vspace{1.5cm}

\noindent
\begingroup
	\Large
	\textbf{LINFO2266: Advanced Algorithms for Optimization}\\\\
	Project 1: Dynamic Programming
\endgroup
\vspace{0.2cm}

\begin{Exercise}[title={Modeling the Traveling Salesman Problem}]

The Traveling Salesman Problem is a famous combinatorial optimization problem where a salesman has to find the shortest tour visiting all cities from a given set $N=\{1,\ldots,n\}$.
The salesman can travel between every pair of cities $i,j \in N$, which are separated by a distance $d_{ij}$.
Find a dynamic programming model for the Traveling Salesman Problem.

\Question What information is contained in the states of the model?
\answerbox{1cm}{
% YOUR ANSWER HERE
}

\Question Give the Bellman recurrence equation that computes the optimal value for each state of the model. You can assume that the salesman starts at city 1.
\answerbox{1cm}{
% YOUR ANSWER HERE
}

\Question Give the base case(s) of the recurrence, and the case that gives the optimal solution of the problem.
\answerbox{1cm}{
% YOUR ANSWER HERE
}

\Question What is the space and time complexity needed by your dynamic proram algorithm for the TSP (wrt the number of cities $n$). Justify?
\answerbox{5cm}{
% YOUR ANSWER HERE
}



\end{Exercise}

\pagebreak

\begin{Exercise}[title={Experiments}]

\Question Create a table with the rune time for all the instances of size 8, 10 and 18 in the data/TSP directory. 
          This table must be generated by a new class with a main method in your repository.
	  Specify the characteristics the machine used for the experiment.
\answerbox{8cm}{
% YOUR ANSWER HERE
}

\Question Create a graphic with the number of instances solved over time, for instances of size 8, 10 and 18.
	  The x axis is the time, the y axis are the number of instances. (you may use log-scale axis for clarity).
	  The graphic can be created from the content of the data in the table.
\answerbox{8cm}{
% YOUR ANSWER HERE
}


\end{Exercise}


\end{document}